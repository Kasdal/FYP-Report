%%
% Copyright (c) 2017 - 2022, Pascal Wagler;
% Copyright (c) 2014 - 2022, John MacFarlane
%
% All rights reserved.
%
% Redistribution and use in source and binary forms, with or without
% modification, are permitted provided that the following conditions
% are met:
%
% - Redistributions of source code must retain the above copyright
% notice, this list of conditions and the following disclaimer.
%
% - Redistributions in binary form must reproduce the above copyright
% notice, this list of conditions and the following disclaimer in the
% documentation and/or other materials provided with the distribution.
%
% - Neither the name of John MacFarlane nor the names of other
% contributors may be used to endorse or promote products derived
% from this software without specific prior written permission.
%
% THIS SOFTWARE IS PROVIDED BY THE COPYRIGHT HOLDERS AND CONTRIBUTORS
% "AS IS" AND ANY EXPRESS OR IMPLIED WARRANTIES, INCLUDING, BUT NOT
% LIMITED TO, THE IMPLIED WARRANTIES OF MERCHANTABILITY AND FITNESS
% FOR A PARTICULAR PURPOSE ARE DISCLAIMED. IN NO EVENT SHALL THE
% COPYRIGHT OWNER OR CONTRIBUTORS BE LIABLE FOR ANY DIRECT, INDIRECT,
% INCIDENTAL, SPECIAL, EXEMPLARY, OR CONSEQUENTIAL DAMAGES (INCLUDING,
% BUT NOT LIMITED TO, PROCUREMENT OF SUBSTITUTE GOODS OR SERVICES;
% LOSS OF USE, DATA, OR PROFITS; OR BUSINESS INTERRUPTION) HOWEVER
% CAUSED AND ON ANY THEORY OF LIABILITY, WHETHER IN CONTRACT, STRICT
% LIABILITY, OR TORT (INCLUDING NEGLIGENCE OR OTHERWISE) ARISING IN
% ANY WAY OUT OF THE USE OF THIS SOFTWARE, EVEN IF ADVISED OF THE
% POSSIBILITY OF SUCH DAMAGE.
%%

%%
% This is the Eisvogel pandoc LaTeX template.
%
% For usage information and examples visit the official GitHub page:
% https://github.com/Wandmalfarbe/pandoc-latex-template
%%

% Options for packages loaded elsewhere
\PassOptionsToPackage{unicode}{hyperref}
\PassOptionsToPackage{hyphens}{url}
\PassOptionsToPackage{dvipsnames,svgnames,x11names,table}{xcolor}
%
\documentclass[
  10pt,
  paper=a4,
  ,captions=tableheading
]{scrartcl}
\usepackage{amsmath,amssymb}
% Use setspace anyway because we change the default line spacing.
% The spacing is changed early to affect the titlepage and the TOC.
\usepackage{setspace}
\setstretch{1.2}
\usepackage{iftex}
\ifPDFTeX
  \usepackage[T1]{fontenc}
  \usepackage[utf8]{inputenc}
  \usepackage{textcomp} % provide euro and other symbols
\else % if luatex or xetex
  \usepackage{unicode-math} % this also loads fontspec
  \defaultfontfeatures{Scale=MatchLowercase}
  \defaultfontfeatures[\rmfamily]{Ligatures=TeX,Scale=1}
\fi
\usepackage{lmodern}
\ifPDFTeX\else
  % xetex/luatex font selection
\fi
% Use upquote if available, for straight quotes in verbatim environments
\IfFileExists{upquote.sty}{\usepackage{upquote}}{}
\IfFileExists{microtype.sty}{% use microtype if available
  \usepackage[]{microtype}
  \UseMicrotypeSet[protrusion]{basicmath} % disable protrusion for tt fonts
}{}
\makeatletter
\@ifundefined{KOMAClassName}{% if non-KOMA class
  \IfFileExists{parskip.sty}{%
    \usepackage{parskip}
  }{% else
    \setlength{\parindent}{0pt}
    \setlength{\parskip}{6pt plus 2pt minus 1pt}}
}{% if KOMA class
  \KOMAoptions{parskip=half}}
\makeatother
\usepackage{xcolor}
\definecolor{default-linkcolor}{HTML}{A50000}
\definecolor{default-filecolor}{HTML}{A50000}
\definecolor{default-citecolor}{HTML}{4077C0}
\definecolor{default-urlcolor}{HTML}{4077C0}
\usepackage[margin=2.5cm,includehead=true,includefoot=true,centering,]{geometry}
\usepackage{etoolbox}
\BeforeBeginEnvironment{lstlisting}{\par\noindent\begin{minipage}{\linewidth}}
\AfterEndEnvironment{lstlisting}{\end{minipage}\par\addvspace{\topskip}}
% add backlinks to footnote references, cf. https://tex.stackexchange.com/questions/302266/make-footnote-clickable-both-ways
\usepackage{footnotebackref}
\setlength{\emergencystretch}{3em} % prevent overfull lines
\providecommand{\tightlist}{%
  \setlength{\itemsep}{0pt}\setlength{\parskip}{0pt}}
\setcounter{secnumdepth}{-\maxdimen} % remove section numbering
\ifLuaTeX
  \usepackage{selnolig}  % disable illegal ligatures
\fi
\IfFileExists{bookmark.sty}{\usepackage{bookmark}}{\usepackage{hyperref}}
\IfFileExists{xurl.sty}{\usepackage{xurl}}{} % add URL line breaks if available
\urlstyle{same}
\hypersetup{
  pdftitle={CI-CD Deployment},
  pdfauthor={Milan Ples - 20088120},
  pdfkeywords={break stuff},
  hidelinks,
  breaklinks=true,
  pdfcreator={LaTeX via pandoc with the Eisvogel template}}
\title{CI-CD Deployment}
\author{Milan Ples - 20088120}
\date{01-01-2023}



%%
%% added
%%

\usepackage[pages=all]{background}

%
% for the background color of the title page
%
\usepackage{pagecolor}
\usepackage{afterpage}
\usepackage{tikz}
\usepackage[margin=2.5cm,includehead=true,includefoot=true,centering]{geometry}

%
% break urls
%
\PassOptionsToPackage{hyphens}{url}

%
% When using babel or polyglossia with biblatex, loading csquotes is recommended
% to ensure that quoted texts are typeset according to the rules of your main language.
%
\usepackage{csquotes}

%
% captions
%
\definecolor{caption-color}{HTML}{777777}
\usepackage[font={stretch=1.2}, textfont={color=caption-color}, position=top, skip=4mm, labelfont=bf, singlelinecheck=false, justification=raggedright]{caption}
\setcapindent{0em}

%
% blockquote
%
\definecolor{blockquote-border}{RGB}{221,221,221}
\definecolor{blockquote-text}{RGB}{119,119,119}
\usepackage{mdframed}
\newmdenv[rightline=false,bottomline=false,topline=false,linewidth=3pt,linecolor=blockquote-border,skipabove=\parskip]{customblockquote}
\renewenvironment{quote}{\begin{customblockquote}\list{}{\rightmargin=0em\leftmargin=0em}%
\item\relax\color{blockquote-text}\ignorespaces}{\unskip\unskip\endlist\end{customblockquote}}

%
% Source Sans Pro as the de­fault font fam­ily
% Source Code Pro for monospace text
%
% 'default' option sets the default
% font family to Source Sans Pro, not \sfdefault.
%
\ifnum 0\ifxetex 1\fi\ifluatex 1\fi=0 % if pdftex
    \usepackage[default]{sourcesanspro}
  \usepackage{sourcecodepro}
  \else % if not pdftex
    \usepackage[default]{sourcesanspro}
  \usepackage{sourcecodepro}

  % XeLaTeX specific adjustments for straight quotes: https://tex.stackexchange.com/a/354887
  % This issue is already fixed (see https://github.com/silkeh/latex-sourcecodepro/pull/5) but the
  % fix is still unreleased.
  % TODO: Remove this workaround when the new version of sourcecodepro is released on CTAN.
  \ifxetex
    \makeatletter
    \defaultfontfeatures[\ttfamily]
      { Numbers   = \sourcecodepro@figurestyle,
        Scale     = \SourceCodePro@scale,
        Extension = .otf }
    \setmonofont
      [ UprightFont    = *-\sourcecodepro@regstyle,
        ItalicFont     = *-\sourcecodepro@regstyle It,
        BoldFont       = *-\sourcecodepro@boldstyle,
        BoldItalicFont = *-\sourcecodepro@boldstyle It ]
      {SourceCodePro}
    \makeatother
  \fi
  \fi

%
% heading color
%
\definecolor{heading-color}{RGB}{40,40,40}
\addtokomafont{section}{\color{heading-color}}
% When using the classes report, scrreprt, book,
% scrbook or memoir, uncomment the following line.
%\addtokomafont{chapter}{\color{heading-color}}

%
% variables for title, author and date
%
\usepackage{titling}
\title{CI-CD Deployment}
\author{Milan Ples - 20088120}
\date{01-01-2023}

%
% tables
%

%
% remove paragraph indention
%
\setlength{\parindent}{0pt}
\setlength{\parskip}{6pt plus 2pt minus 1pt}
\setlength{\emergencystretch}{3em}  % prevent overfull lines

%
%
% Listings
%
%


%
% header and footer
%
\usepackage[headsepline,footsepline]{scrlayer-scrpage}

\newpairofpagestyles{eisvogel-header-footer}{
  \clearpairofpagestyles
  \ihead*{CI-CD Deployment}
  \chead*{}
  \ohead*{01-01-2023}
  \ifoot*{Milan Ples - 20088120}
  \cfoot*{}
  \ofoot*{\thepage}
  \addtokomafont{pageheadfoot}{\upshape}
}
\pagestyle{eisvogel-header-footer}


\backgroundsetup{
scale=1,
color=black,
opacity=0.2,
angle=0,
contents={%
  \includegraphics[width=\paperwidth,height=\paperheight]{background5.pdf}
  }%
}

%%
%% end added
%%

\begin{document}

%%
%% begin titlepage
%%
\begin{titlepage}
\newgeometry{top=2cm, right=4cm, bottom=3cm, left=4cm}
\tikz[remember picture,overlay] \node[inner sep=0pt] at (current page.center){\includegraphics[width=\paperwidth,height=\paperheight]{background5.pdf}};
\newcommand{\colorRule}[3][black]{\textcolor[HTML]{#1}{\rule{#2}{#3}}}
\begin{flushleft}
\noindent
\\[-1em]
\color[HTML]{5F5F5F}
\makebox[0pt][l]{\colorRule[435488]{1.3\textwidth}{4pt}}
\par
\noindent

% The titlepage with a background image has other text spacing and text size
{
  \setstretch{2}
  \vfill
  \vskip -8em
  \noindent {\huge \textbf{\textsf{CI-CD Deployment}}}
    \vskip 2em
  \noindent {\Large \textsf{Milan Ples -
20088120} \vskip 0.6em \textsf{01-01-2023}}
  \vfill
}


\end{flushleft}
\end{titlepage}
\restoregeometry
\pagenumbering{arabic} 

%%
%% end titlepage
%%

% \maketitle


{
\setcounter{tocdepth}{3}
\tableofcontents
\newpage
}
\hypertarget{plagiarism-statement}{%
\section{Plagiarism Statement}\label{plagiarism-statement}}

I confirm that the work contained in this report is my own, except where
due reference is made in the text to the work of others. I have
acknowledged all material, data or ideas from other sources, whether
quoted verbatim or paraphrased, by the use of suitable references and
footnotes. I have also acknowledged any assistance received in preparing
this report.

\hypertarget{acknowledgements}{%
\section{Acknowledgements}\label{acknowledgements}}

I would like to thank my supervisor, Patrick Felicia for his guidance
and support throughout this project. My team at Unum who helped me
immensely during my internship in executing the project and providing
endless support and help.

\hypertarget{abstract}{%
\section{Abstract}\label{abstract}}

The purpose of this project is to streamline the software development
process through the use of a CI-CD workflow that automates the steps of
building, testing, and deploying code changes. This is achieved through
the utilization of git for version control, Github for code repository
management, Jenkins for continuous integration, Docker for packaging
code into containers, and AWS for deployment.

By implementing this workflow, developers can make code changes and have
them automatically built, tested, and deployed, reducing the time and
effort required to bring new features and updates to production. This
enables teams to deliver software updates faster and with fewer errors,
improving the overall efficiency and quality of the software development
process.

\hypertarget{introduction}{%
\section{Introduction}\label{introduction}}

Continuous Integration-Continuous Deployment (CI-CD) is a software
development approach that aims to automate the process of building,
testing, and deploying code changes. This approach is designed to
improve the efficiency and quality of the software development process
by allowing teams to deliver updates and new features faster, with fewer
errors.

One way to implement a CI-CD workflow is through the use of tools such
as git for version control, Github for code repository management,
Jenkins for continuous integration, Docker for packaging code into
containers, and AWS for deployment. In this project, these tools are
used together to create a streamlined process for building, testing, and
deploying code changes. This enables developers to focus on writing code
and adding new features, while the CI-CD pipeline handles the rest.

\hypertarget{purpose-intended-use-and-audience}{%
\subsection{Purpose, Intended Use and
Audience}\label{purpose-intended-use-and-audience}}

The purpose of this CI-CD project is to automate the process of
building, testing, and deploying code changes, in order to improve the
efficiency and quality of the software development process. It is
intended to be used by software development teams who want to streamline
their workflow and deliver updates and new features to their users
faster, with fewer errors.

The intended audience for this project is software developers and devops
professionals who are responsible for building, testing, and deploying
code changes. By implementing a CI-CD workflow, these individuals can
focus on writing code and adding new features, while the pipeline
handles the rest of the process. This allows them to work more
efficiently and effectively, and helps to ensure that code changes are
thoroughly tested and deployed in a timely manner.

Overall, the goal of this CI-CD project is to make the software
development process more efficient, reliable, and scalable, enabling
teams to deliver high-quality software updates to their users more
quickly and with fewer errors.

\hypertarget{rationale-for-the-project}{%
\subsection{Rationale for the Project}\label{rationale-for-the-project}}

As a software developer and Dev-Ops enthusiast, I am well aware of the
benefits of automating processes in order to optimize the development
process and decrease expenses. When I was first introduced to continuous
integration and continuous deployment (CI-CD) during my internship, I
was immediately drawn to the potential this methodology had to
revolutionize how code modifications are created, tested, and deployed.

Utilizing tools such as git, Github, Jenkins, Docker, and AWS, a CI-CD
pipeline can automate the steps of building, testing, and deploying code
changes, decreasing the time and effort needed to bring new features and
updates to production. This not only increases the efficiency of the
development process, but also helps to guarantee the reliability and
quality of the final product.

I have therefore made it a priority to learn about and understand the
implementation of a CI-CD pipeline. I am confident that this approach
has the potential to greatly benefit any organization and its users, and
I am eager to see the positive impact it can have on the software
development process. By embracing automation and adopting a CI-CD
workflow, I believe it is possible to significantly enhance the speed
and reliability of the development process, while also reducing costs
and minimizing errors.

\hypertarget{risk-assessment}{%
\subsection{Risk Assessment}\label{risk-assessment}}

The risks associated with this project are as follows:

As I embark on this project, I am aware that there is a risk that I may
not be able to deliver it as planned. This risk is especially relevant
due to my lack experience with the technologies and tools being used, as
I may encounter difficulties or delays in implementing the desired
functionality. Additionally, if the project is overly complex or
involves a large number of features, it may be difficult for me to
manage and could result in delays or cuting out features.

I also need to consider the possibility that external factors such as
third-party APIs or vendor support may not be reliable or may change
unexpectedly, which could impact the project's timeline and success. To
increase the chances of delivering the project successfully, I will need
to carefully assess and manage these risks. This may involve seeking
additional resources or expertise, breaking the project down into
smaller tasks, or implementing contingency plans. By proactively
addressing these risks, I can increase the likelihood of delivering the
project as planned.

\hypertarget{methodology}{%
\section{Methodology}\label{methodology}}

Waterfall is a traditional linear method of software development that
involves completing each phase of the development process before moving
on to the next, while CICD involves continuously integrating and
deploying code changes.

\hypertarget{introduction-1}{%
\subsection{Introduction:}\label{introduction-1}}

Waterfall and CICD (Continuous Integration and Continuous Deployment)
are two different development methodologies that are used to build
software. While Waterfall is a traditional method that follows a linear,
sequential process, CICD is a more modern approach that emphasizes rapid
iteration and continuous delivery of software updates. In this article,
we will compare the two approaches in terms of cost savings, speed, and
workload.

\hypertarget{cost-savings}{%
\subsection{Cost Savings:}\label{cost-savings}}

One potential advantage of the Waterfall model is that it may be less
expensive to implement, as it does not require the same level of
infrastructure and automation as CICD. Waterfall also typically involves
fewer resources, as teams do not need to constantly monitor and deploy
code changes. However, the Waterfall model can be inflexible and may not
be well-suited to projects with rapidly changing requirements or a high
degree of uncertainty. As a result, teams may need to go back and redo
work that has already been completed, leading to additional costs and
delays.

On the other hand, CICD can offer significant cost savings by reducing
the need for manual testing and deployment processes. By continuously
integrating and deploying code changes, teams can quickly and easily
make updates to their software and respond to changing business needs.
This can also lead to reduced maintenance costs, as teams can more
easily fix defects and make updates to the software as needed. However,
CICD does require a significant investment in infrastructure and
automation, as well as a well-defined process for testing and deploying
code changes.

\hypertarget{speed}{%
\subsection{Speed:}\label{speed}}

The Waterfall model is typically slower than CICD, as each phase of the
software development life cycle (SDLC) must be completed before moving
on to the next phase. This can lead to delays and missed deadlines,
particularly if there are issues or changes that require going back and
redoing work that has already been completed.

In contrast, CICD allows for rapid iteration and continuous delivery of
software updates, allowing teams to quickly respond to changing business
needs or customer feedback. This can lead to faster time-to-market and
the ability to deliver new features and functionality more quickly.
However, it is important to ensure that code changes are thoroughly
tested and do not break existing functionality, as this can lead to
delays and additional work.

\hypertarget{workload}{%
\subsection{Workload:}\label{workload}}

The Waterfall model may require less workload upfront, as teams do not
need to continuously integrate and deploy code changes. However, the
inflexibility of the Waterfall model can lead to additional workload if
changes or defects need to be addressed once a phase has been completed.

CICD, on the other hand, requires a more continuous workload, as teams
must continuously integrate and deploy code changes. This can lead to a
more complex and challenging workload, as teams must ensure that code
changes do not break existing functionality and that the software is
stable and reliable. However, the ability to rapidly iterate and respond
to changing business needs can also lead to a more dynamic and engaging
work environment.

\hypertarget{conclusion}{%
\subsubsection{Conclusion:}\label{conclusion}}

Both Waterfall and CICD have their own benefits and limitations, and the
appropriate approach will depend on the specific needs and constraints
of a project. Waterfall may be more suitable for projects with
well-defined requirements and a low level of uncertainty, while CICD may
be more appropriate for projects with rapidly changing requirements or a
need for rapid iteration and innovation. Ultimately, the choice between
the two approaches will depend on the specific cost, speed, and workload
considerations of the project, as well as the resources and capabilities
of the development team.

\hypertarget{project-development-cycle}{%
\section{Project Development cycle}\label{project-development-cycle}}

I am developing this project with a focus on Agile principles, which
prioritize flexibility and adaptability in the development process. By
embracing an Agile approach, I aim to deliver value to stakeholders in a
timely and efficient manner, while also allowing for rapid iteration and
adaptation to changing requirements.

In order to achieve these goals, I am leveraging a range of tools and
techniques to automate and streamline my development process. These
tools include Docker, Jenkins, and Ansible, which allow me to easily
package, deploy, and manage my application across multiple environments.
By using these tools, I can ensure that my application is reusable,
scalable, and maintainable, and that I can easily and consistently
deliver updates and improvements to stakeholders.

Furthermore, I am adopting a DevOps mindset, which emphasizes
collaboration and integration between development and operations teams.
This allows me to optimize my workflow and continuously deliver value to
stakeholders by automating and streamlining the build, test, and
deployment process. By leveraging the full capabilities of these tools,
I am able to deliver high-quality, reliable software that meets the
needs of stakeholders and users.

\hypertarget{project-design-objectives}{%
\subsection{Project Design Objectives}\label{project-design-objectives}}

In this project, I am separating the flow and design into two distinct
sections: automation of infrastructure and configuration automation.

For the automation of infrastructure, I am using tools such as Terraform
to define and provision my infrastructure resources in a predictable,
version-controlled, and automated manner. These tools allow me to use
code to define and manage my infrastructure resources, such as servers,
networks, and storage, in a repeatable and consistent manner. By using
these tools, I can leverage the benefits of automation and version
control to manage my infrastructure resources in a more efficient and
reliable way.

For configuration automation, I am using tools such as Ansible to
automate the configuration of my resources, such as applications,
services, and environments. These tools allow me to define and enforce
configuration standards, install and configure software, and manage
users and permissions in an automated manner. By using these tools, I
can reduce the risk of errors and improve the efficiency of my workflow,
while also ensuring that my resources are consistently configured
according to my desired standards.

\hypertarget{automated-tests}{%
\subsection{Automated Tests}\label{automated-tests}}

As part of my development process, I am using Jenkins to automate the
execution of my tests. Jenkins is an open-source automation server that
helps developers automate parts of the development process, such as
building, testing, and deploying code changes. It allows me to set up a
series of tasks, known as ``jobs,'' that can be automatically triggered
based on certain events or conditions.

In my project, I am using Jenkins to automatically run my tests whenever
I commit code changes to my repository. This allows me to quickly and
easily validate that my code is working as intended, and identify and
fix any issues early on in the development process. By automating my
tests, I can save time and reduce the risk of errors, ensuring that my
code is of high quality and meets my desired standards.

However, it is important to note that while the use of Jenkins to
automate my tests is a useful aspect of my development process, it is
not the primary focus of my project. There may be other goals and
objectives that are more central to my project, such as developing new
features or improving the performance of the application. Nonetheless,
the use of Jenkins to automate my tests helps me to ensure that my code
is of high quality and meets my desired standards, which is an important
factor in achieving my overall project goals.

\hypertarget{technologies-used}{%
\subsection{Technologies Used}\label{technologies-used}}

\hypertarget{git}{%
\subsubsection{Git}\label{git}}

Git: Git is a version control system that allows developers to track
changes to their codebase and collaborate with other team members. It
allows developers to save different versions of their code and switch
between them as needed, making it easier to track and manage changes.
Git also allows developers to work on the same codebase at the same
time, without having to worry about overwriting each other's work. This
enables teams to work more efficiently and effectively, and helps to
ensure that code changes are thoroughly tested and deployed in a timely
manner.

\hypertarget{github}{%
\subsubsection{Github}\label{github}}

For this project, I will be using GitHub to store the codebase and track
changes through automatic versioning. GitHub is a web-based platform
that allows developers to host and review code, manage projects, and
build software. It is built on top of Git, a version control system that
enables developers to track changes to their code. By using GitHub, I
can easily keep track of the history of my work and make sure that my
code is organized and properly versioned. When changes are made to the
codebase on GitHub, Jenkins will be notified automatically through Git,
allowing me to streamline my workflow and stay up to date on the status
of the project.

\hypertarget{github-actions}{%
\subsubsection{Github Actions}\label{github-actions}}

GitHub Actions is a tool that allows developers to automate their
workflow by setting up a series of tasks, known as ``actions,'' that can
be triggered based on certain events or conditions. These actions can be
used to build, test, and deploy code changes, as well as perform other
tasks such as releasing software and publishing documentation. I am
using GitHub Actions to automate the compilation of my Final Year
Project pdf upon commit to GitHub. This allows me to easily and quickly
generate a compiled version of my project every time I commit code
changes, without having to manually perform the compilation process.

\hypertarget{jenkins}{%
\subsubsection{Jenkins}\label{jenkins}}

Jenkins is an open-source automation server that helps developers
automate parts of the development process, such as building, testing,
and deploying code changes. It allows developers to set up a series of
tasks, known as ``jobs,'' that can be automatically triggered based on
certain events or conditions. For this project, Jenkins will be used to
create a Docker image and upload it to Docker Hub once it detects
changes on GitHub. This allows the project to be easily packaged and
deployed in a containerized environment, making it easier to manage and
scale. Jenkins is a powerful tool for automating development workflows
and can be used in conjunction with other tools, such as GitHub Actions,
to create a complete CI/CD pipeline.

\hypertarget{docker}{%
\subsubsection{Docker}\label{docker}}

In this project, we are using Docker to containerize the application.
Docker is a tool that makes it easier to create, deploy, and run
applications in a containerized environment. A container is a
lightweight, standalone, and executable package that includes everything
an application needs to run, including code, libraries, dependencies,
and runtime. By using Docker, we can easily package and deploy the
application in a consistent manner across different environments. This
allows us to manage and scale the application more easily, as well as
ensure that it runs consistently regardless of the underlying
infrastructure. Widely used saying: ``It works on my machine'' is no
longer valid. Docker allows us to create a consistent environment for
the application, so that it will run the same way on any machine.

\hypertarget{aws}{%
\subsubsection{AWS}\label{aws}}

AWS (Amazon Web Services): AWS is a cloud computing platform that
provides a wide range of services, including computing, storage, and
database management. It allows developers to build, test, and deploy
applications at scale, without having to worry about the underlying
infrastructure. AWS also offers a number of tools and services
specifically designed for continuous integration and continuous
deployment (CI-CD), such as CodePipeline and CodeBuild.

\hypertarget{kubernetes}{%
\subsubsection{Kubernetes}\label{kubernetes}}

Kubernetes: Kubernetes is an open-source container orchestration
platform that allows developers to manage and deploy containerized
applications at scale. It provides features such as automatic
scheduling, self-healing, and horizontal scaling, which make it easier
for developers to deploy and manage applications in a production
environment.

\hypertarget{terraform}{%
\subsubsection{Terraform}\label{terraform}}

Terraform is a tool for building, changing, and versioning
infrastructure safely and efficiently. It is an open-source tool that
allows users to define infrastructure as code (IaC) and manage it using
a command-line interface. With Terraform, users can create, update, and
version infrastructure resources such as virtual machines, networking
components, and storage solutions across multiple cloud providers (such
as AWS, GCP, and Azure) and on-premises data centers.

Using Terraform, users can define infrastructure resources in a
configuration file written in the HashiCorp Configuration Language
(HCL). This configuration file is used to create and manage resources in
a consistent, repeatable way. Users can use Terraform to create, update,
and delete resources in a predictable manner, making it easier to manage
infrastructure changes and roll back updates if necessary.

Terraform also includes features such as dependency management, which
allows users to specify the order in which resources are created, and
resource tracking, which allows users to keep track of resources that
have been created and modified. Overall, Terraform is a powerful tool
for managing infrastructure in a consistent and repeatable way, making
it easier for organizations to build and maintain infrastructure on a
variety of cloud and on-premises platforms.

\hypertarget{ansible}{%
\subsubsection{Ansible}\label{ansible}}

In this project, we are also using Ansible to automate tool
configuration and installation on remote devices. Ansible is a
configuration management and automation tool that allows developers to
easily provision, configure, and manage remote infrastructure. It works
by using simple, human-readable configuration files called ``playbooks''
to define the desired state of the infrastructure and the tasks needed
to achieve it.

In our project, we are using Ansible to automate the configuration and
installation of the application on remote devices. This allows us to
easily and consistently deploy the application to multiple devices
without the need to manually perform the configuration and installation
process on each device. Using Ansible helps us to save time and reduce
the risk of errors by automating these tasks. It also allows us to
easily update and manage the configuration of the application on the
remote devices.

\hypertarget{other-tools-used}{%
\section{Other Tools used}\label{other-tools-used}}

\hypertarget{pandoc}{%
\subsection{Pandoc:}\label{pandoc}}

Pandoc is a tool that I am using in this project to help me convert
files from one format to another. It allows me to easily convert files
between different markup languages, such as Markdown, HTML, and LaTeX.
This is particularly useful for me because it allows me to preserve the
formatting and layout of my documents, even when I need to convert them
to a different format.

\hypertarget{latex}{%
\subsection{LaTeX:}\label{latex}}

LaTeX is a typesetting system that I am using in this project to help me
create high-quality documents with a consistent layout and formatting.
It is particularly useful for documents that contain a lot of
mathematical notation or other complex formatting. By using LaTeX, I can
ensure that my documents look professional and are easy to read, even
when they contain complex formatting.

\hypertarget{discussion}{%
\section{Discussion}\label{discussion}}

\hypertarget{limitations}{%
\subsection{Limitations}\label{limitations}}

\hypertarget{conclusion-1}{%
\section{Conclusion}\label{conclusion-1}}

\hypertarget{references}{%
\section{References}\label{references}}

\begin{itemize}
\tightlist
\item
  Oakland, J. (1987). The Economics of Non-excludability. The Journal of
  Economic Perspectives, 1(1), 19-32. doi:10.1257/jep.1.1.19
\end{itemize}

\end{document}
